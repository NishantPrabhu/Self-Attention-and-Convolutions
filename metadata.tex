% DO NOT EDIT - automatically generated from metadata.yaml

\def \codeURL{https://github.com/NishantPrabhu/Self-Attention-and-Convolutions}
\def \codeDOI{}
\def \dataURL{https://www.cs.toronto.edu/~kriz/cifar.html}
\def \dataDOI{}
\def \editorNAME{}
\def \editorORCID{}
\def \reviewerINAME{}
\def \reviewerIORCID{}
\def \reviewerIINAME{}
\def \reviewerIIORCID{}
\def \dateRECEIVED{03 April 2021}
\def \dateACCEPTED{}
\def \datePUBLISHED{}
\def \articleTITLE{[Re] On the relationship between self-attention and convolutional layers}
\def \articleTYPE{Editorial}
\def \articleDOMAIN{}
\def \articleBIBLIOGRAPHY{bibliography.bib}
\def \articleYEAR{2021}
\def \reviewURL{}
\def \articleABSTRACT{In this report, we perform a detailed study on the paper "On the Relationship between Self-Attention and Convolutional Layers", which provides theoretical and experimental evidence that self attention layers can behave like convolutional layers. The proposed method does not obtain state-of-the-art performance but rather answers an interesting question - \textit{do self-attention layers process images in a similar manner to convolutional layers?} This has inspired many recent works which propose fully-attentional models for image recognition. We focus on experimentally validating the claims of the original paper and our inferences from the results led us to propose a new variant of the attention operation - {\em Hierarchical Attention}. The proposed method shows significantly improved performance with fewer parameters, hence validating our hypothesis. To facilitate further study, all the code used in our experiments are publicly available here\footnote{\url{https://github.com/NishantPrabhu/Self-Attention-and-Convolutions}}.}
\def \replicationCITE{}
\def \replicationBIB{}
\def \replicationURL{}
\def \replicationDOI{}
\def \contactNAME{Nishant Prabhu}
\def \contactEMAIL{me17b084@smail.iitm.ac.in}
\def \articleKEYWORDS{rescience c, rescience x}
\def \journalNAME{ReScience C}
\def \journalVOLUME{4}
\def \journalISSUE{1}
\def \articleNUMBER{}
\def \articleDOI{}
\def \authorsFULL{Mukund Varma T and Nishant Prabhu}
\def \authorsABBRV{M. Varma and N. Prabhu}
\def \authorsSHORT{Mukund and Nishant}
\title{\articleTITLE}
\date{}
\author[1]{Mukund Varma T}
\author[1]{Nishant Prabhu}
\affil[1]{Indian Institute of Technology, Madras, India}
